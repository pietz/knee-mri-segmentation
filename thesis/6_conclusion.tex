\section{Conclusion}

In this thesis, I presented the development of a fully automated workflow based on convolutional neural networks, which segments bones in three dimensional MRI data of human knees. It shows an excellent performance of 98\% DSC while distinguishing between Femur, Tibia, and Fibula. The network even shows accurate results on sagittal images although it was entirely trained using coronal data. It is applicable to changes in crop and resolution due to its fully convolutional structure. This also helps to visualize all of the layers in the network. Furthermore, I was able to combine the masked bone images with part of the architecture as a pre-trained model to assess the age of the candidates with an MAE of less than 200 days. All of this is possible using non-invasive magnetic resonance imaging.

The segmentation results show improved performance when compared to other studies and further prove the importance of neural networks in the field of medical imaging. Although processing speed was no priority, the architecture consists of only 210,000 parameters making it 150 times smaller than U-Net. On a sub-\$1000 workstation, it can be trained from scratch in 1 to 4 hours depending if the bones need to be separated. Using the pre-trained weights, one can likely apply transfer learning with minimal effort to solve similar problems.

Creating multiple ground truth segmentations by hand and averaging the results, could present a way of improving the findings in this thesis. This would further reduce the noise in the data and allow the model to move beyond the 98\% mark. Since the smallest architecture achieved the highest scores, the challenge of improving the performance is likely not limited by the capacity of the network.

In conclusion, this thesis focussed on the segmentation of bones, but only briefly addressed the age assessment based on the resulting data. It is to be investigated more thoroughly how the segmentation maps can be used to further improve the results on age related estimates. Additional information about the patients like height and weight may help to improve accuracy in the future.

\newpage