\section{Introduction}

\subsection{Field and Context}

Recent advances in artificial intelligence have led to fully automated workflows that often exceed human performance. State of the art neural networks can classify images into thousands of categories more accurate and magnitudes faster than we can \cite{He2015a}. They translate text between hundreds of languages \cite{Wu2016}, navigate cars autonomously through cities \cite{Bojarski2017} and detect malware in computer systems \cite{Saxe2015}. In most of these cases they have been trained on tens of thousands or even millions of data samples to reach their accuracy. 

Neural networks have also found great success in the medical field, where data sets are often much smaller. Although the same techniques can be applied, they need to be adjusted to match a particular problem. In many applications neural networks have been setting the current state of the art when it comes to different segmentation or classification tasks.

\subsection{Research Problem}

The age assessment is a challenging procedure to determine the chronological age of a person lacking legal documents. It exists to afford children provisions they are entitled to by law and to prevent adults from taking unwarranted advantage of these benefits. There is no method that offers an exact identification of the age, but several assessments are currently used that can be separated in two groups \cite{EuropeanAsylumSupportOffice2013}.

Personal interviews or examinations are held to gain an observation of physical or psychological features. These practices include a high amount of manual work and they are also subjective to the person conducting the test. The other type of assessments use X-rays to observe physical traits like the collar bone, teeth or carpal growth plates. Besides that these images are also analyzed manually comes the problem that X-rays are considered an invasive imaging method because they require the exposure of ionizing radiation \cite{EuropeanAsylumSupportOffice2013}.

As such, there is a high demand for a fully automated, unprejudiced and non-invasive method for accurately determining the age of a person.

Based on this idea, I was introduced to a data set of MRIs showing the right knee of several candidates. In contrast to CT and X-ray, MRI is considered non-invasive. Similar to the carpal analysis, the three bones around the knee show growth plates which have been used as indicators for the age of a person. Due to the underlying technology used in MRI machines, these images show a high amount of detail in non-bone tissue. Despite this being an advantage is many other medical applications, it adds complex information outside the scope of the bone and growth plates.

Neural networks are known to be feature selectors, meaning that they will learn to extract information that are relevant to the task \cite{Setiono1997}. This assumes that the size of the data set and the complexity of the problem enable the network to find correlating features. Based on a series of tests, I was not able to create a stable algorithm that would predict the age of a candidate based on their knee MRI. A logical next step towards this goal was removing non-bone tissue from the MR images to reduce the complexity of the data samples.

\subsection{Previous Research}

A study led by Dodin et al. in 2011 focussed on the same goal by masking the Femur and Tibia bone from other soft tissue. They used the ray casting technique, which disassembles the MR images into several surface layers to find the boundaries of the bones. Afterward, multiple partial maps were merged for the final result \cite{Dodin2011}.

Dam et al. presented another approach in 2015 focussing on the segmentation of cartilages in the knee for the research on osteoarthritis. Their method combines the use a multiatlas rigid registration and voxel classification. Besides masking medial and lateral cartilages, they also applied their technique on the Tibia.

\subsection{Focus of this Thesis}

The focus of this thesis revolves around creating a fully automated workflow that segments the long bones in 3D MRIs of human knees. Convolutional neural networks will be the base tool for this study because they have been setting state of the art results in a broad majority of image segmentation tasks. In the end I will present a proof of concept by using the segmented data to estimate the age of candidates.

\newpage
