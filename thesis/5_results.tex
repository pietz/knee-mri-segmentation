\section{Results}

The results in this section are entirely based on the test set, which had no contact with the network up to this point. It was also made sure that there was no overlap in images of people that were recorded multiple times. 

5 images were chosen randomly from each of the two sources in the very beginning. Since the Epi data featured 41 slices and the Jopp data featured 24 slices, 325 2D samples were available for evaluation. Each slice contained 50,176 pixel values, making a total of 16.3 million predictions that were analyzed.

The network architecture was developed and optimized with focus on a single segmentation channel that merged Femur, Tibia und Fibula maps. However, tests were also run to validate the performance for each bone on its own. The same architecture and training procedure was used for this task.

\subsection{Numeric Evaluation}

The proposed model achieves a DICE score of 98.0\% and an IoU of 96.0\%. Precision and Recall are perfectly balanced at 98.0\% as well, suggesting that predictions are neither too optimistic or pessimistic. The error shows a small value of 1.2\% when looking at falsely predicted areas in relationship to the entire frame.

\begin{table}[H]
    \centering
    \begin{tabular}{| l | c | c | c | c | c |}
    \hline
           & DSC & IoU & Precision & Recall & Error \\ 
    \hline
    Merged   & \makecell{0.980} 
             & \makecell{0.960} 
             & \makecell{0.980} 
             & \makecell{0.980} 
             & \makecell{0.012} \\
    \hline
    Femur    & \makecell{0.981} 
             & \makecell{0.963} 
             & \makecell{0.979} 
             & \makecell{0.984} 
             & \makecell{0.006} \\
    \hline
    Tibia    & \makecell{0.977} 
             & \makecell{0.955} 
             & \makecell{0.976} 
             & \makecell{0.977} 
             & \makecell{0.006} \\
    \hline
    Fibula   & \makecell{0.953} 
             & \makecell{0.911} 
             & \makecell{0.954} 
             & \makecell{0.952} 
             & \makecell{0.001} \\
    \hline
    Combined & \makecell{0.979} 
             & \makecell{0.958} 
             & \makecell{0.977} 
             & \makecell{0.981} 
             & \makecell{0.004} \\
    \hline
    \end{tabular}
    \caption{Numeric evaluation of the test set using popular metrics}
\end{table}

Results on Femur and Tibia alone are very comparable to the merged approach, whereas the Fibula segmentation reaches a DSC of 95.3\% and IoU of 91.1\%. This could be due to the fact that the Fibula is only visible in a minority of slices, making it a fairly unbalanced task.

Combining the three separate segmentations to a single model gives comparable results as well. The error is even reduced by a factor of 3, which is expected because the segmentation channels are also increased by 3.

A study from 2011 ran a similar segmentation on the knee \cite{Martel-Pelletier2011}, achieving a DSC of 94\% for the femur and 92\% for the tibia using the ray casting technique. Another study from 2015  \cite{Dam} used an atlas based segmentation and achieved a DSC of 97.5\% for the tibia.

\subsection{Visual Evaluation}

\newpage