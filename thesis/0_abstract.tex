\renewcommand{\abstractname}{\large Abstract}
\begin{abstract}

\vspace{0.5cm}

The age assessment is a complicated procedure used to determine the chronological age of an individual who lacks legal documentation. While there is no method that provides an exact identification of the age, current practices require high amounts of manual work and invasive X-ray imaging. Previous publications show that the condition of growth plates in the knee could be an appropriate indicator for the border to adulthood. As part of a DFG study, noninvasive MRI recordings were collected to investigate this hypothesis further. This thesis provides a step towards a solution by automating the extraction of bone in knee MRIs to reduce the data complexity. The segmentations show 98\% DSC using convolutional neural networks and further improve the work of previous studies. In the end, this data is used in a proof of concept to assess the age of individuals with a mean difference of 0.55 years.

\end{abstract}
\newpage